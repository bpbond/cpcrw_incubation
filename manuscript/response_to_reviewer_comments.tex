% Response to referee comments, with reference to changed points in text
% Doing this in LaTeX seemed easiest since the response to reviewers doc
% was already set up that way...but I wish Biogeosciences would support Markdown
% BBL October 2016
% TODO means haven't addressed yet; LINES means addressed, need to fill in line #s

\documentclass[11pt, oneside]{article}
\usepackage[utf8]{inputenc}     
\usepackage{geometry}
\usepackage{url}
\geometry{letterpaper} 
\usepackage{graphicx}		

\begin{document}
% ======================================================================================
\subsection*{Response to Referee \#1}

{\it 1.3. I think that the work would benefit from a more thorough comparison with boreal forest incubations across the Arctic. }

This point was raised by Referee \#2 as well (comment 2.4). We now provide a more complete comparison, citing for example studies such as the Schädel et al. (2016) meta-analysis, Dutta et al. (2006), Lavoie et al. (2011), Karhu et al. (2010), and Wickland and Neff (2008). {\bf See new lines 308-337, 368-372, 401-408, 427-432.}

\medskip
{\it 1.4. N section would benefit from more Arctic-centric comparisons of N limitations and in particular of boreal forest N dynamics. Q10 can be temperature dependent, also depending on N limitation in the system. }

We have supplemented this section with a better comparison to relevant literature, for example, Lavoie et al. (2011), Sistla et al. (2012), and Bouskill et al. (2014). {\bf See new lines 391-408.}

\medskip
{\it 1.5. This study raises interesting questions. In mineral soils, under woody vegetation that might be of low C quality, and slower C pool, one might expect higher temperature sensitivity. I think that these questions, even if not addressed directly by the data presented, should have been discussed more explicitly. Comparison with other Arctic woody plant systems would be instructive. }

This is similar to the referee's comment 1.9 (please see our response to that below), with the added factor of C quality. We have addressed this more explicitly in our revision, referring for example to incubation studies on this question (Fierer et al. 2005). {\bf See new lines 414-420.}

\medskip
{\it 1.6. Studies have shown that moisture can have a weaker effect on temperature sensitivity early on during an incubation experiment, in the presence of more labile C. This relative to the effect on moisture on the Q-10 of cumulative respiration, reflecting slow turning over C - this could be an interesting analysis to include here, and would help to assess how }

This is an interesting suggestion. We do not observe any evidence for changes in CO2 moisture sensitivity with time, and weak changes in CO2 temperature sensitivity; CH4 emissions show a weak decline in moisture sensitivity with time. This is now discussed in the text. {\bf See new lines 260-261, 366-371.}

\medskip
{\it 1.7. How do your results in terms of temperature and moisture sensitivity (especially under drought conditions) scale with Alaskan climate change predictions from modelers? How does it compare with deep soils incubations (mineral soils) from the Arctic, and from boreal ecosystems? }

The first question is similar to a point raised by Referee \#2 (comment 2.3). We have added this as a paragraph in the discussion, noting e.g. observable anthropogenic influences on high-latitude precipitation, drier and warmer conditions in boreal Eurasia, and growing season length increases in interior Alaska with no increase in precipitation. {\bf See new lines 326-337.}

The second question largely repeats, we think, both referees' suggestions to better compare our results to previous work, in particular boreal and Arctic incubations; see our responses to 1.3 and 1.4 above.

\medskip
{\it 1.8. Line 31-34: I cannot find discussion of this point in the rest of the text, and while important, this statement is relatively vague and there are no cited references. Since it underpins the rational for studying deep, unfrozen Arctic soils, it would be helpful to expand on this more in the manuscript. }

Referee \#2 raised this point as well (comment 2.6), and it's a good one. We now better describe why deep active-layer soils, such as those studied here, and important and distinctive relative to permafrost or shallow active layer soils.  {\bf See new lines 87-98.}

\medskip
{\it 1.9. Lines 48-60: I think that this section would benefit from an introduction of the interactions between the specific ecosystem (upland boreal forest) you are studying, and its interaction with soil chemistry, since vegetation type is influential in terms of soil carbon quality and quantity. Woody plant biomass tends to have a higher C:N ratio relative to herbaceous dominated systems, and this tends to result in lower quality resources for microbial communities. }

Vegetation and ecosystem type is a significant factor that is not well explored here, we agree. We have added some points about this in the introduction and discussion. {\bf See new lines 51-63, 433-439.}

\medskip
{\it 1.10. Lines 70-72: These are really important considerations, and it seems appropriate to discuss them more explicitly. How is the temperature and precipitation regime of the boreal forest of interior AK expected to change? There are also indirect effects of vegetation type on soil temperatures that could be discussed here. }

We agree, although this largely echoes comments 1.7 and 1.9; please see our responses above.

\medskip
{\it 1.11. Lines 72-74: While these are important questions, they are not really addressed in this study, and so either it might make sense to leave it out, or to discuss the particulars as they apply to this study, ie: the importance and questions related to C:N ratios. }

Referee 3 made this point as well, and this sentence has been removed.

\medskip
{\it 1.12. Lines 77-80: I think a stronger argument for why deep active-layer soils can be made, and it would be helpful to clarify what are the 'strong effects' of warming. }

Agreed. See our response to comment 1.8 above.

\medskip
{\it 1.13. I cannot tell if C:N, \%C and \%N were measured at the end of the incubation. Could these results be collated in a table in the manuscript? Otherwise the methods section appears to be detailed and well written. }

C and N were measured for all samples post-incubation, and in the 'extra' group (l. 128-129) pre-incubation. This will be clarified in the methods, particularly lines 162-, which we agree were ambiguous. A new Table 1 now summarizes a variety of physical and flux data by treatment. Reviewer 2 also raised the idea of looking at C/N, and we have responded to that suggestion in detail (see comment 2.9).

\medskip
{\it 1.14. Line 232: In this section it would also be interesting to know the soil respiration decay rate per treatment over the course of the incubation experiment. }

This has been included in the new Table 1 (see comment 1.13 above).

\medskip
{\it 1.15. Line 238-240: Confusingly worded sentence. }

This has been clarified. {\bf See new lines 264-266.}

\medskip
{\it 1.16. I don't think that the summary of nearby respiration studies add very much to the discussion section. Perhaps if the similarities and discrepancies were more integral to the central findings of the paper or integrated differently into the discussion they would seem more meaningful here. Perhaps comparing with other boreal incubations (eg: Lee et al., 2012; Lavoie et al., 2011) would help to provide some additional context. }

We agree that the comparison to other boreal respiration studies needs improvement, and this echoes Referee \#2's comment 2.4. First, although we think the paragraph about nearby studies provides useful context, we have tightened it considerably. Second, we have restructured and improved the subsequent comparison section, discussing a variety of studies suggested by all the reviewers to better put our results in the context of previous work. {\bf See new lines 295-307 for nearby studies, and expanded context in lines 308-337, 368-372, 401-408, 427-432.}

\medskip
{\it 1.17. Line 270: There is missing punctuation after the word 'results'. }

This will be fixed. {\bf See new line 298.}

\medskip
{\it 1.18. Line 286-293: Perhaps the new synthesis by Schadel et al., 2016, would also be a useful comparison here. }

This point was also made by Referee \#2 (comment 2.2). The fact that we didn't cite the Schädel et al. (2016) meta-analysis was a quirk of timing, as it appeared after our manuscript was submitted. In our revision, we have significantly expanded this paragraph, discussing and comparing to Schädel et al. (2016) in depth, particularly their findings of higher aerobic than anaerobic respiration; respiration dominance of CO2 versus CH4; and Q10 values. We also cite and discuss a variety of other studies. {\bf See new lines 46, 81, 313, 368-371, 411.}

\medskip
{\it 1.19. Line 293: That soil moisture may be as important a control on microbial respiration as temperature is an important finding in recent incubation studies, and the potential to define its interaction with temperature will help modelers of soil decomposition better constrain the physical parameters of microbial respiration rates. This feels buried in the manuscript, and I think that it would improve the paper if it were highlighted better throughout the text. }

Thanks for the useful suggestion; this point is now brought out more clearly. {\bf See new lines 333-337, 356-365, 454-458.}

\medskip
{\it 1.20. Line 311-317: This section could be better explained in the context of the discussion or omitted altogether. It seems less important to defend the plausibility of relatively low temperature sensitivity, but instead to try to explain it in the context of these soil characteristics. Could low temperature sensitivity be the result of low C quality in this deep soil environment? }

We appreciate this useful advice and question. We have diminished the emphasis on defending this finding, and instead try to place it in the context of soil characteristics in this mixed-species boreal forest, SOC quality, etc. {\bf See new lines 366-390.}

\medskip
{\it 1.21. Line 322-332: This section, which lays out the crux of the paper, the interaction between temperature and moisture sensitivity in driving microbial respiration is relatively vague. It would be good to describe the less-temperature-sensitive processes that would be important to consider for more stable-C metabolism. And how does moisture play a role here? Perhaps DOC becomes more limiting in the drought conditions? }

This is interesting to consider: what mechanisms might produce a Q10 {\it increase} under drought conditions? This is opposite to what is usually observed (e.g. Jassal et al. 2008), but the field is rife with contradictory results (von Lützow and Kögel-Knabner 2009). We have made this paragraph more specific in this area. {\bf See new lines 378-390.}

\medskip
{\it 1.22. Line 356: The Janssens et al., 2010, citation refers to a meta-analysis of temperate forest soils that are not nitrogen limited. There are studies focusing on Arctic N cycling that would be more appropriate, and many Arctic studies have shown that N availability can limit C mineralization rates. Is this site considered to be N limited in the deep active layer? }

We agree that referring and comparing to studies such as Lavoie et al. (2011) and Bouskill et al. (2014), which focused specifically on high latitudes, would be a useful addition. We don't know of any studies examining the N limitation of deep soils at this site. We now discuss these results in our revision, along with other studies examining the relationship between N availability and C mineralization. {\bf See new lines 392-408.}

\medskip
{\it 1.23. Line 367: Is this comparison, with North American soils, relevant to this study? }

It's true that Colman and Schimel (2014) include only a few studies that could be termed boreal (from Maine, USA). We have removed this comparison.

\medskip
{\it 1.24. Line 383-384: Can you be more explicit in your meaning here? How do you mean that there is weakness in what can be inferred about temperature sensitivity from experiments? }

We basically meant what the title of the Podrebarac et al. (2016) paper says: "Soils isolated during incubation underestimate temperature sensitivity of respiration and its response to climate history". I.e., incubation soils are isolated from their natural environment, and as a result we need to be cautious about extrapolating incubation results to {\it in situ} responses. We have clarified this in the text. {\bf See new lines 425-426.}


% ======================================================================================
\newpage
\subsection*{Response to Referee \#2}

{\it 2.2. It seems like this paper was published as a discussion paper before Schädel et al. 2016 was published and hence a discussion of the meta-analysis was not possible but should be addressed in the revisions. }

This point was also made by Referee \#1 (comment 1.18). Yes, the fact that we didn't cite the Schädel et al. (2016) meta-analysis was a quirk of timing, as it appeared after our manuscript was submitted. We have significantly expanded the discussion on this point, comparing our results to Schädel et al. (2016) in depth, particularly their findings of higher aerobic than anaerobic respiration; respiration dominance of CO2 versus CH4; and Q10 values. {\bf See new lines 46, 81, 313, 368-371, 411.}

\medskip
{\it 2.3. The importance of the results would be more obvious if the discussion also contained an upscaling or circumpolar aspect of drought in the Arctic. It would be useful to have some discussion about the area that is expected to be most affected by drought. This is important as changes in temperature will affect most of the Arctic, whereas drought effects or dry soils will occur more locally. }

We have added this as a paragraph in the discussion, noting e.g. observable anthropogenic influences on high-latitude precipitation, drier and warmer conditions in boreal Eurasia, and growing season length increases in interior Alaska with no increase in precipitation. {\bf See new lines 326-337.}

\medskip
{\it 2.4. 1) Throughout the manuscript, I have noticed that important papers from the permafrost literature are missing. This applies to C stocks in the permafrost area, Tarnocai et al. 2009 is a good paper but there are more recent and more accurate estimates of permafrost C stocks described in Hugelius et al. 2014 and Schuur et al. 2015 that should be cited. When it comes to the permafrost C feedback, Schuur et al. 2015 is currently the best and most up to date review. In addition, Koven et al. 2015 is a good one too. The discussion on incubation literature should include papers like Lavoie et al. 2011, Dutta et al. 2006, and Schädel et al. 2014. }

We appreciate the referee drawing our attention to these omissions. While Schuur et al. (2015) is already cited, Hugelius (2014) and Koven (2015; though we do cite his 2011 paper) are useful additions. The Lavoie paper is very useful with respect to N and microbial respiration, while Dutta et al. (2006), although it concerns Siberian soils, is also a good comparison. We had not included Schädel et al. (2014) simply because of its focus on permafrost, versus the active-layer soils studied here, but we agree it is also be a reasonable addition. All these references are now cited throughout the manuscript. {\bf See new lines 308-337, 368-372, 401-408, 427-432.}

\medskip
{\it 2.5. 2) L. 31: Permafrost thaws and does not melt }

Fixed. {\bf See new line 33.}

\medskip
{\it 2.6. 3) A better explanation is needed why deep-active layer soils are different to active layer or permafrost soils, I couldn't find a strong argument for why they would behave differently. Also, deep-active layer soils are those that are the most impacted by inter annual variability in thaw depth and so they might switch between active layer in one year to permafrost in another, that's worth some discussion as well. }

This is a good point. We now better describe why deep active-layer soils, such as those studied here, are important and distinctive relative to permafrost or shallow active layer soils. {\bf See new lines 87-98.}

\medskip
{\it 2.7. 4) The statistics in this paper are generally good and I would like to compliment the authors on making the entire data set and analysis available online. I would still suggest that the manuscript would profit from some additional details on collinearity of the tested variables as well as model outputs such as AIC. }

Thank you. We appreciate the useful suggestions, and now provide these additional details in our revised manuscript. {\bf See new lines 257-270.}

\medskip
{\it 2.8. 5) Add a table with soil properties such as bulk density, \%C etc. }

This useful suggestion was also made by Referee \#1 (comment 1.13). We have done so, in a new Table 1.

\medskip
{\it 2.9. 6) Why not include C/N as a variable in the statistical analysis? Schädel et al. 2014 showed that C/N is a good predictor of C release and can be used as a scaling factor. It would be interesting to see if C release from short-term incubations show the same result }

This is an interesting suggestion. We added code (see \url{https://github.com/bpbond/cpcrw_incubation/commit/426a91e1bbd21200718b334d3295fbef40a1ea6}) to compute C/N and examine its significance as a predictor. Currently C/N seems to be a poorer predictor than \%N. We now discuss this issue, referencing previous work such as Schädel et al (2014). {\bf See new lines 409-420.}

\medskip
{\it 2.10. 7) In the discussion, it would be good to also include the warming potential of CO2 and CH4 especially when making assumptions about the permafrost C feedback, it is briefly mentioned in line 348 but a more in depth discussion would be good }

That's a very good point-thank you-and integrates well with an expanded comparison to the Schädel et al. (2016) paper (cf. comment 2.2 above) and other publications (comment 2.4 above). {\bf See new lines 350-354.}

\medskip
{\it 2.11. 8) the conclusions might be a bit strong given the data and previous results published }

We have added caveats, noting in particular the useful but incremental nature of this study. {\bf See new lines 449-460.}


% ======================================================================================
\newpage
\subsection*{Response to Referee \#3}

{\it 3.2. My main criticism is that I think that the authors over-emphasize the results of the daily emissions and that the authors should further explore (or report) the results of the controls of the cumulative C emissions. I'm curious as to whether the relationships with soil C/N and \%N observed in daily emissions still hold on cumulative emissions. The comparison between these soil parameters (i.e. ones that probably don't change much throughout the course of the incubation, including temperature) and the cumulative fluxes is perhaps more appropriate. Perhaps modelers find the controls on daily fluxes interesting and these are likely quite useful in regards to the relationship between moisture and C production (i.e. changes on a daily basis), but I think that the controls on cumulative fluxes are quite interesting and could be further explored. }

We agree that rebalancing the manuscript, focusing a bit more on controls on cumulative emissions and a bit less on the instantaneous fluxes, would strengthen it. Accordingly, we now more fully explore controls on the cumulative emissions, and have moved the table summarizing the instantaneous CH4 flux model, to an appendix. {\bf See new lines 826-832.}

\medskip
{\it 3.3. For example, how do the results of soil properties vs. emissions compare to those of Schädel et al. (2014) and Schädel et al. (2016)? How do the moisture results compare to those of Wickland et al. (2008)? }

The other referees both mentioned this as well. The fact that we didn't cite the Schädel et al. (2016) meta-analysis was a quirk of timing, as it appeared after our manuscript was submitted. We have significantly expanded this, discussing and comparing to Schädel et al. (2014, 2016) and Wickland et al. (2008). {\bf See new lines 46, 81, 313, 368-371, 411, and line 306 for Wickland.}

\medskip
{\it 3.4. I do think that the time series of fluxes could be moved to the supplemental materials if the cumulative fluxes are explored in greater detail. I think this paper could be shortened a little bit although I didn't find the length of the paper onerous. Along these lines, I think that the results summarized above from the cumulative emissions should be included in the abstract. }

We have moved one table to supplementary material (see response to comment 3.2 above), and now summarize cumulative emissions results in the abstract. {\bf See new lines 27-31.}

\medskip
{\it 3.5. 22: Daily CO2 fluxes?
26: positive or negative correlation?
27: daily CH4 flux?
28: cumulative production as CO2-as CH4. }

These points have been clarified, except for the last, as we feel it's already clear and unambiguous. {\bf See new lines 20-36.}

\medskip
{\it 3.6. 29: Not really sure how the comparison as to the relative controls of T and moisture are evaluated. }

This statement has been reworded to remove the comparison. {\bf See new lines 31-33.}

\medskip
{\it 3.7. 50: see also updates in Hugelius et al. (2014)
63: Under some conditions (Olefeldt et al 2013): vague and confusing. Please clarify. }

Reviewer 2 also raised the issue (comment 2.4) of our incomplete citation of relevant literature. The Tarnocai reference has been replaced by one to Hugelius et al. (2014), and the Olefeldt sentence clarified. {\bf See new lines 52 and 74.}

\medskip
{\it 3.8. 67: 'substantial variabilities between studies' WHY? }

We have expanded on this point, pointing out that such variability originates from factors such as differences in soil type, antecedent conditions, phase changes, experimental protocols, etc. {\bf See new lines 75-77.}

\medskip
{\it 3.9. 72: Yes, this is an important question, but given that this isn't measured in this study, perhaps this sentence should be omitted or re-written. }

This sentence has been removed.

\medskip
{\it 3.10. 101: When did sampling occur?
112: Specify at the time of sampling
140: How frequently was moisture adjusted? Requires a bit more explanation. Were instantaneous moisture values used in analysis? }

Sampling date is reported in line 110. We have clarified 80 cm at the time of sampling. Moisture adjustment was done after every mass measurement, i.e. every timepoint shown in Figure 1; this has been clarified. {\bf See new lines 164-165.}

\medskip
{\it 3.11. 211: Please remember to complete DOI }

Done. {\bf See new lines 234-235.}

\medskip
{\it 3.12. 215: Not sure what this value for soil dry mass indicates }

It's just useful, we think, to give readers a good sense of sample size.

\medskip
{\it 3.13. 216: Standard deviation for \%C and \%N is nearly 100\%. Check values. }

Thanks. There was a great of variability (obviously), but distributed throughout the data set-i.e., this isn't driven by one or two outliers.

\medskip
{\it 3.14. 229: add units
231: add units
233: positively correlated
241: positively correlated
245-246: 106 }

These have all been fixed. {\bf See new lines 252-270.}

\medskip
{\it 3.15. 253-254: So what variables were significant in predicting cumulative C emissions? }

Please see our response to comment 3.2 above.

\medskip
{\it 3.16. 262: First mention of vegetation stress, remove, not clear how it's related. }

We now better integrate this point, mentioning it in the introduction and clarifying its relationship to the study goals. {\bf See new lines 58-62, 291, 310-315.}

\medskip
{\it 3.17. 270. Add '.' }

This has been fixed. {\bf See new line 298.}

\medskip
{\it 3.18. 271: Specify soil type in which these measurements were made (results not surprising for a forest soil) }

Upland Cryosols; we have clarified this. {\bf See new line 298.}

\medskip
{\it 3.19. 272: What about results from Wickland et al. (2008). Study found threshold for moisture importance
305-307: again, see Wickland et al. (2008) }

Please see our response to comment 3.7 above.

\medskip
{\it 3.20. 322-324: cool! }

Agreed!

\medskip
{\it 3.21. 344-345: Specify that the results in Treat et al. (2015) were for anaerobic incubations and were thus likely to be much smaller. }

Thanks; we have done so. {\bf See new line 321.}

\medskip
{\it 3.22. 347-348: See also Lee et al. (2012)
364-365: See also Schadel et al. (2014). Also, I thought this section was a bit vague, probably could be shortened slightly. }

Thanks for the Lee et al. reference, which we had not considered (see our response to 3.7 above) but is now cited. We have also reworked and tightened section 4.2. {\bf See new lines 354, 391-420.}

\medskip
{\it 3.23. 383-384: 'specific weaknesses': vague
384: See also lag effects found in Treat et al. (2015) }

This awkward language has been removed, and a note about lag effects added. {\bf See new lines 425-429.}

\medskip
{\it 3.24. 393: 'taking them out of depth' rephrase. Also could use this argument for the section on CH4 production. }

We have reworded this. {\bf See new lines 440-447.}

\medskip
{\it 3.25. Fig.1 : Edit figure to be color-blind friendly. }

We thought we were already doing so in using the {\tt RColorBrewer} package, not the default palette of {\tt ggplot2}, but have shifted to using a color-blind friendly palette from \url{http://www.cookbook-r.com/Graphs/Colors_(ggplot2)/#a-colorblind-friendly-palette} in all figures.

\medskip
{\it 3.26. Fig. 2,3: When did watering / moisture adjustment occur? Consider indicating with arrows and specifying in text. }

Moisture adjustment was done after every mass measurement, i.e. every timepoint shown in Figure 1. This has been clarified. {\bf See new lines 164-165.}

\medskip
{\it 3.27. Fig. 4: Switch top and bottom panels as CO2 is always discussed before CH4. Also edit colors and patterns to be color-blind friendly. }

Good point-fixed. Re colors, see our response to 3.25 above.

\end{document}